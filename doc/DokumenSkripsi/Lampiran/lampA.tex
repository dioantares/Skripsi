%versi 3 (18-12-2016)
\chapter{Kode Program Formulir Data Umat}
\label{lamp:A}

%terdapat 2 cara untuk memasukkan kode program
% 1. menggunakan perintah \lstinputlisting (kode program ditempatkan di folder yang sama dengan file ini)
% 2. menggunakan environment lstlisting (kode program dituliskan di dalam file ini)
% Perhatikan contoh yang diberikan!!
%
% untuk keduanya, ada parameter yang harus diisi:
% - language: bahasa dari kode program (pilihan: Java, C, C++, PHP, Matlab, C#, HTML, R, Python, SQL, dll)
% - caption: nama file dari kode program yang akan ditampilkan di dokumen akhir
%
% Perhatian: Abaikan warning tentang textasteriskcentered!!
%


\lstinputlisting[language=html, label ={lampA:index.html},  caption= Kode pada \texttt{index.html}]{./Lampiran/index.html} 

\lstinputlisting[language=html, label ={lampA:script.js},  caption= Kode pada \texttt{script.js}]{./Lampiran/script.js}

\lstinputlisting[language=html, label ={lampA:style.css},  caption= Kode pada \texttt{style.css}]{./Lampiran/style.css}

