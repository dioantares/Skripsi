\chapter{Kesimpulan dan Saran}
\label{chap:kesimpulandansaran}

\section{Kesimpulan}
\label{sec:kesimpulan}

\subsection{Kesimpulan Formulir Data Umat}
\label{sec:6:kesimpulanFormulir}

Dengan memanfaatkan fitur yang sudah ada dan menambahkan fitur baru, halaman Formulir Data Umat dapat berfungsi sebagai formulir online untuk pengisian data umat baru. Berikut ini adalah fitur-fitur yang diimplementasikan:
	
\begin{itemize}
	
	\item Halaman formulir dapat dibuka di mobile dengan baik \textit{(responsive design)}.
	\item Memunculkan keyboard yang tepat untuk input tertentu (contoh: nomor telepon menggunakan keypad)
	\item Menyimpan data secara otomatis di penyimpanan lokal, sehingga saat dibuka kembali, umat dapat melanjutkan pengisian. Fitur ini telah diimplementasikan pada tombol \textit{save} dan \textit{load}.
	\item Membangkitkan kode QR untuk nantinya dibaca Odoo.
\end{itemize}

\section{Saran}
\label{sec:saran}
Berdasarkan hasil pengembangan yang dilakukan, berikut adalah saran-saran untuk pengembangan selanjutnya:

\begin{itemize}
	\item Membangkitkan kode QR yang lebih mudah dipindai oleh QR Scanner.
	\item Odoo dapat memindai kode QR, dan menyajikannya bersebelahan (side-by-side) dengan data yang sudah tercatat sebelumnya
\end{itemize}

