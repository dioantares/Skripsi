\chapter{Kesimpulan dan Saran}
\label{chap:kesimpulandansaran}

\section{Kesimpulan}
\label{sec:kesimpulan}

\subsection{Kesimpulan Formulir Data Umat dan Odoo}
\label{sec:6:kesimpulanFormulir}

Dengan memanfaatkan fitur yang sudah ada dan menambahkan fitur baru, halaman Formulir Data Umat dapat berfungsi sebagai formulir online untuk pengisian data umat baru, dan membuat halaman custome module Odoo. Berikut ini adalah fitur-fitur yang diimplementasikan:
	
\begin{itemize}
	
	\item Membuat proses pencatatan data tidak perlu dilakukan secara manual dengan cara membuat halaman formulir website yang dapat dibuka di mobile dengan baik \textit{(responsive design)}.
	\item Data yang telah dituliskan oleh umat dapat dipindai oleh sistem SIMU dengan cara menyimpan data secara otomatis di penyimpanan lokal, sehingga saat dibuka kembali, umat dapat melanjutkan pengisian. Fitur ini telah diimplementasikan pada tombol \textit{save} dan \textit{load}, selanjutnya terdapat fitur tombol \textit{submit} yang berfungsi untuk membangkitkan kode QR untuk nantinya dibaca Odoo.
	\item Belum berhasil membuat QR Scanner pada halaman Odoo yang sudah dikonfigurasi oleh penulis.
\end{itemize}

\section{Saran}
\label{sec:saran}
Berdasarkan hasil pengembangan yang dilakukan, berikut adalah saran-saran untuk pengembangan selanjutnya:

\begin{itemize}
	\item Odoo dapat memindai kode QR, dan menyajikannya bersebelahan (side-by-side) dengan data yang sudah tercatat sebelumnya.
	\item Membeli plugin berbayar yang sudah tersedia di Odoo.
\end{itemize}

