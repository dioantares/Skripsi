%versi 2 (8-10-2016) 
\chapter{Pendahuluan}
\label{chap:intro}
   
\section{Latar Belakang}
\label{sec:label}

Pada saat ini kebutuhan manusia terhadap teknologi sangatlah tinggi, dapat dilihiat dalam kehidupan sehari-hari manusia tidak terlepas dari penggunaan alat teknologi, karena dalam penggunaan teknologi dapat berfungsi sebagai alat untuk mempermudah melakukan sesuatu. Kemajuan teknologi yang kian pesat pada era modern ini membawa berbagai dampak pada banyak aspek kehidupan, termasuk dalam suatu organisasi. Sistem Informasi Manajemen Umat (SIMU) adalah aplikasi milik Keuskupan Bandung, aplikasi ini bertujuan untuk mencatat data umat dan dinamikanya (contohnya adalah sakramen). Keuskupan Bandung memiliki sekitar 108.000 umat, plus umat Sibolga.

Dengan banyaknya jumlah umat yang terdapat dalam sistem informasi dan tidak menutup kemungkinan akan terus bertambah, maka akan dibuat sebuah sistem dengan memanfaatkan perkembangan teknologi pada saat ini. Salah satunya adalah judul skripsi penulis pada saat ini yaitu Pemanfaatan QR Code dalam Input Data Odoo, Studi Kasus: SIMU. Pemanfaatan QR Code ini bertujuan untuk mempermudah, mempercepat proses input data dan mengurangi kesalahan penulisan dalam input data, karena data yang diinput sudah berdasarkan penulisan umat itu sendiri.

Sebelum sistem ini dibuat, maka jika perlu ada data umat yang dimasukkan ke sistem informasi manajemen umat (SIMU), prosedurnya adalah sebagai berikut:
\begin{enumerate}
	\item Admin paroki memberikan blanko formulir data umat kepada umat.
	\item Umat mengisikan datanya ke dalam formulir tersebut secara tertulis.
	\item Formulir dikembalikan kepada admin paroki. 
	\item Admin paroki mengetikkan data yang dituliskan di atas formulir.
\end{enumerate}
Prosedur ini membutuhkan waktu yang lama dan kurang efisien, admin paroki memiliki kemungkinan untuk melakukan kesalahan dalam proses input data, karena admin paroki perlu untuk membaca ulang dan mengetikkan kembali data yang dituliskan di atas formulir kedalam sistem input data.

Pada skripsi ini yang berjudul Pemanfaatan QR Code dalam Input Data Odoo, Studi Kasus: SIMU, akan dibuat sebuah sistem yang dapat melakukan input data secara otomatis memanfaatkan QR Code pada input data Odoo, sehingga program yang dibuat ini akan menjadi dua aplikasi utama yaitu membuat halaman html sederhana (website) yang berisi form SIMU dan membuat sistem Odoo yang berisi data field yang menyerupai data umat SIMU dan sistem yang mampu memindai QR Code. Quick Response Code (QR Code) merupakan gambar dua dimensi yang memiliki kemampuan untuk menyimpan data. QR Code biasa digunakan untuk menyimpan data berupa teks, baik itu numerik, alfanumerik, maupun kode biner \cite{qrcode:13:median}.

Program ini bertujuan agar umat dan admin paroki dapat lebih mudah dalam pengisian formulir, skripsi ini akan dirancang menggunakan Odoo. Odoo adalah aplikasi Enterprise Resource Planning open source berbasis Bahasa Python. Enterprise Resource Planning (ERP) adalah sebuah sistem informasi terintegrasi yang dapat mengakomodasi kebutuhan–kebutuhan informasi secara spesifik yang ada di perusahaan. Odoo adalah rangkaian aplikasi bisnis open source yang mencakup banyak kebutuhan, beberapa diantaranya adalah eCommerce, akuntansi, inventaris dan manajemen proyek.

\section{Rumusan Masalah}
\label{sec:rumusan}
Rumusan masalah yang akan dibahas di skripsi ini adalah sebagai berikut :
\begin{enumerate}
	\item Bagaimana proses pencatatan data agar dapat lebih efisien?
	\item Bagaimana agar pencatatan data dapat dipindai oleh sistem SIMU?
	\item Bagaimana agar umat tidak perlu melakukan penulisan tangan namun dapat diakses ke SIMU?
\end{enumerate} 


\section{Tujuan}
\label{sec:tujuan}
Tujuan yang ingin dicapai dari penulisan skripsi ini sebagai berikut :
\begin{enumerate}
	\item Membangun halaman HTML yang yang responsif (terbaca mudah di ponsel) dan berisikan formulir sistem informasi manajemen umat (SIMU).
	\item Membangkitkan kode QR berdasarkan data yang telah diisi untuk nantinya dibaca oleh Odoo.
	\item Membangun sistem Odoo yang berisi data yang field-fieldnya menyerupai data umat dan sistem mampu memindai kode QR dari halaman formulir yang telah diisi oleh umat.
\end{enumerate}


\section{Batasan Masalah}
\label{sec:batasan}
Beberapa batasan yang dibuat terkait dengan pengerjaan skripsi ini adalah sebagai berikut :
\begin{enumerate}
	\item Sistem yang dibuat bukan untuk mempercepat proses pengisian data, namun berfungsi untuk meningkatkan efisiensi proses pencatatan data, karena tidak diperlukan kembali penulisan tangan dan penyalinan data.
\end{enumerate}


\section{Metodologi}
\label{sec:metlit}
Metodologi yang dilakukan pada skripsi ini adalah sebagai berikut :
\begin{enumerate}
	\item Melakukan studi literatur pembuatan modul Odoo.
	\item Melakukan studi literatur User Experience yang baik di Mobile.
	\item Menganalisis ukuran dan jumlah QR Code yang dibutuhkan.
	\item Membangun sistem yang dapat dibuka di mobile dengan baik (responsive design), memunculkan keyboard yang tepat untuk input tertentu (contoh: nomor telepon menggunakan keypad), dan menyimpan data secara otomatis di penyimpanan lokal, sehingga saat dibuka kembali, umat dapat melanjukan pengisian.
	\item Melakukan pengujian dan eksperimen.
	\item Menulis dokumen skripsi.
\end{enumerate}


\section{Sistematika Pembahasan}
\label{sec:sispem}
Sistematika penulisan setiap bab skripsi ini adalah sebagai berikut :
\begin{itemize}
	\item Bab 1 Pendahuluan \\
	Membahas latar belakang, rumusan masalah, tujuan, batasan masalah, metodologi, dan sistematika pembahasan.
	\item Bab 2 Landasan Teori \\
	Membahas teori-teori yang berhubungan dengan penelitian ini, yaitu Odoo, QR Code, Python
	\item Bab 3 Analisis \\
	Membahas analisis terhadap sistem Odoo dan SIMU.
	\item Bab 4 Perancangan \\
	Membahas perancangan fitur yang akan diimplementasikan pada halaman website formulir dan SIMU.
	\item Bab 5 Implementasi dan Pengujian \\
	Membahas implementasi fitur Odoo pada studi kasus SIMU dan pengujian yang dilakukan.
	\item Bab 6 Kesimpulan dan Saran \\
	Membahas kesimpulan dari penelitian ini dan saran untuk penelitian berikutnya.
\end{itemize}
