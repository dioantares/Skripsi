%versi 2 (8-10-2016) 
\chapter{Pendahuluan}
\label{chap:intro}
   
\section{Latar Belakang}
\label{sec:label}

Pada saat ini kebutuhan manusia terhadap teknologi sangatlah tinggi, dapat dilihiat dalam kehidupan sehari-hari manusia tidak terlepas dari penggunaan alat teknologi, karena dalam penggunaan teknologi dapat berfungsi sebagai alat untuk mempermudah melakukan sesuatu. Kemajuan teknologi yang kian pesat pada era modern ini membawa berbagai dampak pada banyak aspek kehidupan, termasuk dalam suatu organisasi. Sistem Informasi Manajemen Umat (SIMU) adalah aplikasi milik Keuskupan Bandung, aplikasi ini bertujuan untuk mencatat data umat dan dinamikanya (contohnya adalah sakramen). Keuskupan Bandung memiliki sekitar 108.000 umat, plus umat Sibolga.

Dengan banyaknya jumlah umat yang terdapat dalam sistem informasi dan tidak menutup kemungkinan akan terus bertambah, maka akan dibuat sebuah sistem dengan memanfaatkan perkembangan teknologi pada saat ini. Salah satunya adalah judul skripsi penulis pada saat ini yaitu Pemanfaatan QR Code dalam Input Data Odoo, Studi Kasus: SIMU. Pemanfaatan QR Code ini bertujuan untuk mempermudah, mempercepat proses input data dan mengurangi kesalahan penulisan dalam input data, karena data yang diinput sudah berdasarkan penulisan umat itu sendiri.

Sebelum sistem ini dibuat, maka jika perlu ada data umat yang dimasukkan ke sistem informasi manajemen umat (SIMU), prosedurnya adalah sebagai berikut:
\begin{enumerate}
	\item Admin paroki memberikan blanko formulir data umat kepada umat.
	\item Umat mengisikan datanya ke dalam formulir tersebut secara tertulis.
	\item Formulir dikembalikan kepada admin paroki. 
	\item Admin paroki mengetikkan data yang dituliskan di atas formulir.
\end{enumerate}
Prosedur ini membutuhkan waktu yang lama dan kurang efisien, admin paroki memiliki kemungkinan untuk melakukan kesalahan dalam proses input data, karena admin paroki perlu untuk membaca ulang dan mengetikkan kembali data yang dituliskan di atas formulir kedalam sistem input data.

Pada skripsi ini yang berjudul Pemanfaatan QR Code dalam Input Data Odoo, Studi Kasus: SIMU, akan dibuat sebuah sistem yang dapat melakukan input data secara otomatis memanfaatkan QR Code pada input data Odoo, sehingga program yang dibuat ini akan menjadi dua aplikasi utama yaitu membuat halaman html sederhana (website) yang berisi form SIMU dan membuat sistem Odoo yang berisi data field yang menyerupai data umat SIMU dan sistem yang mampu memindai QR Code.

Program ini bertujuan agar umat dan admin paroki dapat lebih mudah dalam pengisian formulir, skripsi ini akan dirancang menggunakan Odoo. Odoo adalah aplikasi Enterprise Resource Planning open source berbasis Bahasa Python. 

QR Code (Quick Response Code) merupakan gambar dua dimensi yang memiliki kemampuan untuk menyimpan data. QR Code biasa digunakan untuk menyimpan data berupa teks, baik itu numerik, alfanumerik, maupun kode biner \cite{qrcode:13:median}.





\section{Rumusan Masalah}
\label{sec:rumusan}
Bagian ini akan diisi dengan penajaman dari masalah-masalah yang sudah diidentifikasi di bagian sebelumnya. 


\section{Tujuan}
\label{sec:tujuan}
Akan dipaparkan secara lebih terperinci dan tersturkur apa yang menjadi tujuan pembuatan template skripsi ini


\section{Batasan Masalah}
\label{sec:batasan}
Untuk mempermudah pembuatan template ini, tentu ada hal-hal yang harus dibatasi, misalnya saja bahwa template ini bukan berupa style \LaTeX{} pada umumnya (dengan alasannya karena belum mampu jika diminta membuat seperti itu)


\section{Metodologi}
\label{sec:metlit}
Tentunya akan diisi dengan metodologi yang serius sehingga templatenya terkesan lebih serius.


\section{Sistematika Pembahasan}
\label{sec:sispem}
Rencananya Bab 2 akan berisi petunjuk penggunaan template dan dasar-dasar \LaTeX.
Mungkin bab 3,4,5 dapt diisi oleh ketiga jurusan, misalnya peraturan dasar skripsi atau pedoman penulisan, tentu jika berkenan.
Bab 6 akan diisi dengan kesimpulan, bahwa membuat template ini ternyata sungguh menghabiskan banyak waktu.
