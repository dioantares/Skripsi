\chapter{Landasan Teori}
\label{sec:landasanteori}

\section{Odoo}
\label{sec:odoo}

Odoo adalah aplikasi \textit{Enterprise Resource Planning} (ERP) open source adalah web aplikasi yang dibangun menggunakan bahasa pemrograman Phyton, XML, dan JavaScript dan menggunakan PostgreSQL sebagai database management sistemnya.  Odoo merupakan sebuah sistem atau software manajemen open source, yang sangat mudah untuk digunakan. Bentuk dari sistem Odoo ini terdapat berbagai macam, diantaranya adalah berbasis web, desktop serta mobile. Selain itu, software ini memiliki banyak kelebihan seperti didukung oleh banyak komunitas, modul yang lengkap dan terintegrasi, pemasangan yang mudah, dan juga biaya yang terjangkau. Aplikasi bisnis yang terintegrasi dalam Odoo berbentuk modul-modul yang siap untuk diunduh dan digunakan dan sebagian besar bisa didapatkan secara gratis \cite{suminten}.

\subsection{Struktur Direktori}
\label{sec:strukturDirektori}

\subsection{Instalasi}
\label{sec:instalasi}

\section{Sistem Informasi Manajemen Umat (SIMU)}
\label{sec:simu}
Sistem Informasi Manajemen Umat (SIMU) adalah aplikasi milik Keuskupan Bandung, aplikasi ini bertujuan untuk mencatat data umat dan dinamikanya (contohnya adalah sakramen). Keuskupan Bandung memiliki sekitar 108.000 umat, plus umat Sibolga. Cara kerja sistem ini adalah apabila terdapat ada umat baru yang sebelumnya tidak tercatat di SIMU, berikan print-out dari Formulir Data Umat kepada yang bersangkutan. Jika keluarga juga belum tercatat di SIMU, berikan pula print-out dari Formulir Keluarga Katolik / Rumah Tangga Katolik untuk diisi. Formulir ini biasanya dimiliki oleh paroki masing-masing. Jika tidak tersedia, bisa menghubungi admin keuskupan untuk mendapatkannya.


\section{Design untuk Aplikasi Mobile}
\label{sec:designMobility}
Perangkat seluler (smartphone), tablet, perangkat yang dapat dikenakan, perangkat game genggam telah menjadi umum di dunia komputasi. Desain seluler mencakup aktivitas teknis dan nonteknis yang meliputi beberapa hal, yaitu menetapkan tampilan dan nuansa aplikasi seluler (termasuk aplikasi seluler, WebApps, realitas virtual (VR), dan game), membuat tata letak estetika antarmuka pengguna, menetapkan ritme interaksi pengguna, mendefinisikan struktur arsitektur keseluruhan, mengembangkan konten dan fungsionalitas yang berada di dalam arsitektur, dan merencanakan navigasi yang terjadi di dalam produk seluler. 

Desain seluler biasa dilakukan oleh software engineers, graphic designers, content developers, security specialists, dan semua tim yang tergabung dalam pembuatan model design. Desain sangatlah penting karena memungkinkan suatu model yang dibuat dapat meningkat nilai kualitasnya. Salah satu contohnya adalah website. 

Website adalah sekumpulan halaman yang terdiri dari beberapa laman yang berisi informasi dalam bentuk data digital baik berupa text, gambar, video, audio, dan animasi lainnya yang disediakan melalui jalur koneksi internet. Untuk membangun sebuah halaman website dibutuhkan sebuah bahasa pemrograman yang lebih dikenal dengan sebutan web scripting. 


\subsection{QR Code}
QR Code, kependekan dari Quick Response Code, merupakan gambar dua dimensi yang memiliki kemampuan untuk menyimpan data. QR Code biasa digunakan untuk menyimpan data berupa teks, baik itu numerik, alfanumerik, maupun kode biner. QR Code banyak digunakan untuk keperluan komersil biasanya berisi link url ke alamat tertentu atau sekedar teks berisi iklan, promosi, dan lain-lain. QR Code adalah image dua dimensi yang merepresentasikan
suatu data, terutama data berbentuk teks. QR Code merupakan evolusi dari barcode yang awalnya satu dimensi menjadi dua dimensi. QR Code memiliki kemampuan menyimpan data yang
lebih jauh besar daripada barcode.