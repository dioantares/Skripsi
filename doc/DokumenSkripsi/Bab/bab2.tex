\chapter{Landasan Teori}
\label{sec:landasanteori}

\section{Odoo}
\label{sec:odoo}

Odoo adalah aplikasi \textit{Enterprise Resource Planning} (ERP) open source adalah web aplikasi yang dibangun menggunakan bahasa pemrograman Phyton, XML, dan JavaScript dan menggunakan PostgreSQL sebagai database management sistemnya.  Odoo merupakan sebuah sistem atau software manajemen open source, yang sangat mudah untuk digunakan. Bentuk dari sistem Odoo ini terdapat berbagai macam, diantaranya adalah berbasis web, desktop serta mobile. Selain itu, software ini memiliki banyak kelebihan seperti didukung oleh banyak komunitas, modul yang lengkap dan terintegrasi, pemasangan yang mudah, dan juga biaya yang terjangkau. Aplikasi bisnis yang terintegrasi dalam Odoo berbentuk modul-modul yang siap untuk diunduh dan digunakan dan sebagian besar bisa didapatkan secara gratis \cite{suminten}.

\subsection{Enterprise Resource Planning (ERP)}
\label{sec:erp}
\textit{Enterprise Resource Planning} (ERP) adalah perencanaan sumber daya perusahaan dan merupakan sebuah sistem informasi yang digunakan oleh sebuah perusahaan barang atau jasa yang berguna untuk mengintegrasikan semua proses jalannya perusahaan dari segala aspek baik proses produksi, operasional, distribusi, dan proses lainnya dari produk atau jasa dari perusahaan tersebut. ERP dirancang agar dapat mengkoordinasikan semua sumber daya, informasi dan aktivitas yang diperlukan untuk proses bisnis perusahaan 	\cite{putra}.

\subsection{Instalasi}
\label{sec:instalasi}
Untuk menjalankan Odoo, diperlukan sebuah server Linux yang memenuhi prasyarat berikut ini:

\section{Sistem Informasi Manajemen Umat (SIMU)}
\label{sec:simu}

\section{QR Code}
\label{sec:qr}

\section{Python}
\label{sec:python}