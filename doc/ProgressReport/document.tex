\documentclass[a4paper,twoside]{article}
\usepackage[T1]{fontenc}
\usepackage[bahasa]{babel}
\usepackage{graphicx}
\usepackage{graphics}
\usepackage{float}
\usepackage[cm]{fullpage}
\pagestyle{myheadings}
\usepackage{etoolbox}
\usepackage{setspace} 
\usepackage{lipsum} 
\setlength{\headsep}{30pt}
\usepackage[inner=2cm,outer=2.5cm,top=2.5cm,bottom=2cm]{geometry} %margin
% \pagestyle{empty}

\makeatletter
\renewcommand{\@maketitle} {\begin{center} {\LARGE \textbf{ \textsc{\@title}} \par} \bigskip {\large \textbf{\textsc{\@author}} }\end{center} }
\renewcommand{\thispagestyle}[1]{}
\markright{\textbf{\textsc{Laporan Perkembangan Pengerjaan Skripsi\textemdash Sem. Genap 2023/2024}}}

\onehalfspacing
 
\begin{document}

\title{\@judultopik}
\author{\nama \textendash \@npm} 

%ISILAH DATA BERIKUT INI:
\newcommand{\nama}{Dio Antares}
\newcommand{\@npm}{2017730003}
\newcommand{\tanggal}{16/06/2023} %Tanggal pembuatan dokumen
\newcommand{\@judultopik}{Pemanfaatan QR Code dalam Input Data Odoo, Studi Kasus: SIMU} % Judul/topik anda
\newcommand{\kodetopik}{PAN:5493}
\newcommand{\jumpemb}{1} % Jumlah pembimbing, 1 atau 2
\newcommand{\pembA}{Pascal Alfadian, Nugroho, M.Comp.}
\newcommand{\pembB}{-}
\newcommand{\semesterPertama}{52 - Ganjil 23/24} % semester pertama kali topik diambil, angka 1 dimulai dari sem Ganjil 96/97
\newcommand{\lamaSkripsi}{1} % Jumlah semester untuk mengerjakan skripsi s.d. dokumen ini dibuat
\newcommand{\kulPertama}{Skripsi 2} % Kuliah dimana topik ini diambil pertama kali
\newcommand{\tipePR}{C} % tipe progress report :
% A : dokumen pendukung untuk pengambilan ke-2 di Skripsi 1
% B : dokumen untuk reviewer pada presentasi dan review Skripsi 1
% C : dokumen pendukung untuk pengambilan ke-2 di Skripsi 2

% Dokumen hasil template ini harus dicetak bolak-balik !!!!

\maketitle

\pagenumbering{arabic}

\section{Data Skripsi} %TIDAK PERLU MENGUBAH BAGIAN INI !!!
Pembimbing utama/tunggal: {\bf \pembA}\\
Pembimbing pendamping: {\bf \pembB}\\
Kode Topik : {\bf \kodetopik}\\
Topik ini sudah dikerjakan selama : {\bf \lamaSkripsi} semester\\
Pengambilan pertama kali topik ini pada : Semester {\bf \semesterPertama} \\
Pengambilan pertama kali topik ini di kuliah : {\bf \kulPertama} \\
Tipe Laporan : {\bf \tipePR} -
\ifdefstring{\tipePR}{A}{
			Dokumen pendukung untuk {\BF pengambilan ke-2 di Skripsi 1} }
		{
		\ifdefstring{\tipePR}{B} {
				Dokumen untuk reviewer pada presentasi dan {\bf review Skripsi 1}}
			{	Dokumen pendukung untuk {\bf pengambilan ke-2 di Skripsi 2}}
		}
		
\section{Latar Belakang}
Pada saat ini kebutuhan manusia terhadap teknologi sangatlah tinggi, dapat dilihiat dalam kehidupan sehari-hari manusia tidak terlepas dari penggunaan alat teknologi, karena dalam penggunaan teknologi dapat berfungsi sebagai alat untuk mempermudah melakukan sesuatu. Kemajuan teknologi yang kian pesat pada era modern ini membawa berbagai dampak pada banyak aspek kehidupan, termasuk dalam suatu organisasi. Sistem Informasi Manajemen Umat (SIMU) adalah aplikasi milik Keuskupan Bandung, aplikasi ini bertujuan untuk mencatat data umat dan dinamikanya (contohnya adalah sakramen). Keuskupan Bandung memiliki sekitar 108.000 umat, plus umat Sibolga.

Dengan banyaknya jumlah umat yang terdapat dalam sistem informasi dan tidak menutup kemungkinan akan terus bertambah, maka akan dibuat sebuah sistem dengan memanfaatkan perkembangan teknologi pada saat ini. Salah satunya adalah judul skripsi penulis pada saat ini yaitu Pemanfaatan QR Code dalam Input Data Odoo, Studi Kasus: SIMU. Pemanfaatan QR Code ini bertujuan untuk mempermudah, mempercepat proses input data dan mengurangi kesalahan penulisan dalam input data, karena data yang diinput sudah berdasarkan penulisan umat itu sendiri.

Sebelum sistem ini dibuat, maka jika perlu ada data umat yang dimasukkan ke sistem informasi manajemen umat (SIMU), prosedurnya adalah sebagai berikut:
\begin{enumerate}
	\item Admin paroki memberikan blanko formulir data umat kepada umat.
	\item Umat mengisikan datanya ke dalam formulir tersebut secara tertulis.
	\item Formulir dikembalikan kepada admin paroki. 
	\item Admin paroki mengetikkan data yang dituliskan di atas formulir.
\end{enumerate}
Prosedur ini membutuhkan waktu yang lama dan kurang efisien, admin paroki memiliki kemungkinan untuk melakukan kesalahan dalam proses input data, karena admin paroki perlu untuk membaca ulang dan mengetikkan kembali data yang dituliskan di atas formulir kedalam sistem input data.

Pada skripsi ini yang berjudul Pemanfaatan QR Code dalam Input Data Odoo, Studi Kasus: SIMU, akan dibuat sebuah sistem yang dapat melakukan input data secara otomatis memanfaatkan QR Code pada input data Odoo, sehingga program yang dibuat ini akan menjadi dua aplikasi utama yaitu membuat halaman html sederhana (website) yang berisi form SIMU dan membuat sistem Odoo yang berisi data field yang menyerupai data umat SIMU dan sistem yang mampu memindai QR Code. Quick Response Code (QR Code) merupakan gambar dua dimensi yang memiliki kemampuan untuk menyimpan data. QR Code biasa digunakan untuk menyimpan data berupa teks, baik itu numerik, alfanumerik, maupun kode biner \cite{qrcode:13:median}.

Program ini bertujuan agar umat dan admin paroki dapat lebih mudah dalam pengisian formulir, skripsi ini akan dirancang menggunakan Odoo. Odoo adalah aplikasi Enterprise Resource Planning open source berbasis Bahasa Python. Enterprise Resource Planning (ERP) adalah sebuah sistem informasi terintegrasi yang dapat mengakomodasi kebutuhan–kebutuhan informasi secara spesifik yang ada di perusahaan. Odoo adalah rangkaian aplikasi bisnis open source yang mencakup banyak kebutuhan, beberapa diantaranya adalah eCommerce, akuntansi, inventaris dan manajemen proyek.


\section{Rumusan Masalah}
Rumusan masalah yang akan dibahas di skripsi ini adalah sebagai berikut :
\begin{enumerate}
	\item Bagaimana membuat proses pencatatan data tidak perlu dilakukan secara manual?
	\item Bagaimana agar data yang telah dituliskan oleh umat dapat dipindai oleh sistem simulasi SIMU?
	\item Bagaimana agar data yang telah dituliskan oleh umat dapat diakses ke sistem simulasi SIMU?
\end{enumerate}

\section{Tujuan}
Tujuan yang ingin dicapai dari penulisan skripsi ini sebagai berikut :
\begin{enumerate}
	\item Membangun halaman HTML yang yang responsif (terbaca mudah di ponsel) dan berisikan formulir sistem informasi manajemen umat (SIMU).
	\item Membangkitkan kode QR berdasarkan data yang telah diisi untuk nantinya dibaca oleh Odoo.
	\item Membangun sistem Odoo yang berisi data yang field-fieldnya menyerupai data umat dan sistem mampu memindai kode QR dari halaman formulir yang telah diisi oleh umat.
\end{enumerate}.

\section{Detail Perkembangan Pengerjaan Skripsi}
Detail bagian pekerjaan skripsi sesuai dengan rencan kerja atau laporan perkembangan terkahir :
	\begin{enumerate}
		\item \textbf{Melakukan studi mengenai komponen yang diperlukan untuk membuat Pemanfaatan QR Code dalam Input Data Odoo, Studi Kasus : SIMU}\\
		{\bf Status :} Ada sejak rencana kerja skripsi.\\
		{\bf Hasil :} Studi mengenai komponen yang diperlukan untuk membuat Pemanfaatan QR Code dalam Input Data Odoo, Studi Kasus : SIMU, yaitu dengan membuat, membangun halaman HTML yang yang responsif (terbaca mudah di ponsel) dan berisikan formulir sistem informasi manajemen umat (SIMU), membangkitkan kode QR berdasarkan data yang telah diisi untuk nantinya dibaca oleh Odoo, membangun sistem Odoo yang berisi data yang field-fieldnya menyerupai data umat dan sistem mampu memindai kode QR dari halaman formulir yang telah diisi oleh umat.
		
		Ketiga poin diatas merupakan poin utama dari komponen yang diperlukan untuk membuat Pemanfaatan QR Code dalam Input Data Odoo, Studi Kasus : SIMU, dalam proses pembuatanya, harus ditentukan ukuran QR Code yang akan digunakan (agar dapat dipindai), library yang akan digunakan, dan masih banyak lagi. Sehingga proses studi pada bagian ini masih belum selesai.
		
		\item \textbf{Mempelajari struktur Odoo}\\
		{\bf Status :} Ada sejak rencana kerja skripsi.\\
		{\bf Hasil :} Studi mengenai struktur Odoo dilakukan dengan membaca dan mempelajari dokumentasi pada halaman resmi Odoo. Odoo adalah aplikasi \textit{Enterprise Resource Planning} (ERP) open source adalah web aplikasi yang dibangun menggunakan bahasa pemrograman Python, XML, dan JavaScript dan menggunakan PostgreSQL sebagai database management sistemnya.  Odoo merupakan sebuah sistem atau software manajemen open source, yang sangat mudah untuk digunakan. Bentuk dari sistem Odoo ini terdapat berbagai macam, diantaranya adalah berbasis web, desktop serta mobile. Selain itu, software ini memiliki banyak kelebihan seperti didukung oleh banyak komunitas, modul yang lengkap dan terintegrasi, pemasangan yang mudah, dan juga biaya yang terjangkau. Aplikasi bisnis yang terintegrasi dalam Odoo berbentuk modul-modul yang siap untuk diunduh dan digunakan dan sebagian besar bisa didapatkan secara gratis

		\item \textbf{Membuat halaman HTML berisikan form SIMU dan menampilkan QR Code untuk dipindai.}\\
		{\bf Status :} Ada sejak rencana kerja skripsi.\\
		{\bf Hasil :} Membuat halaman HTML berisikan form SIMU dan menampilkan QR Code untuk dipindai sudah dibuat, namun masih terdapat error pada beberapa bagian, terutama pada bagian QR Code. Halaman website sudah dibuat responsif, pada bagian ini sudah hampir selesai, tinggal bagian menyempurnakan saja.

		\item \textbf{Merancang sistem Odoo untuk SIMU}\\
		{\bf Status :} Ada sejak rencana kerja skripsi.\\
		{\bf Hasil :} Merancang sistem Odoo untuk SIMU sudah dilakukan, proses yang sudah dilakukan adalah, penulis berhasil instalasi Odoo pada perangkat yang digunakan, lalu penulis juga sudah berhasil melalukan custom module (membuat modul custom), namun pada bagian custom module ini belum dilakukan penyempuraan, sehingga sistem Odoo untuk SIMU belum bisa ditampilkan.

		\item \textbf{Melakukan pengujian dan eksperimen}\\
		{\bf Status :} Ada sejak rencana kerja skripsi.\\
		{\bf Hasil :} Pada pengerjaan skripsi ini sudah dilakukan pengujian fungsional dan eksperimen, namun baru dilakukan di halaman html yang dibuat. Pada pengerjaan ini, masih terdapat beberapa kendala yang harus diperbaiki.

		\item \textbf{ Menulis bab 1-6 dokumen skripsi}\\
		{\bf Status :} Ada sejak rencana kerja skripsi.\\
		{\bf Hasil :} Dokumen skripsi sudah ditulis sampai Bab 1, dan Bab 2. Pada bagian Bab 3, baru saja akan dilakukan penulisan, karena waktu selama satu semester ini dipergunakan untuk membuat program.

	\end{enumerate}

\section{Pencapaian Rencana Kerja}
Langkah-langkah kerja yang berhasil diselesaikan dalam Skripsi 2 ini adalah sebagai berikut:
\begin{enumerate}
\item Melakukan studi mengenai komponen yang diperlukan untuk membuat Pemanfaatan QR Code dalam Input Data Odoo, Studi Kasus : SIMU
\item Mempelajari struktur Odoo
\item Membuat halaman HTML berisikan form SIMU dan menampilkan QR Code untuk dipindai.
\item Merancang sistem Odoo untuk SIMU
\item Melakukan pengujian dan eksperimen
\item Menulis dokukemn skripsi
\end{enumerate}



\section{Kendala yang Dihadapi}
%TULISKAN BAGIAN INI JIKA DOKUMEN ANDA TIPE A ATAU C
Kendala - kendala yang dihadapi selama mengerjakan skripsi :
\begin{itemize}
	\item Terlalu banyak menunda waktu, sehingga kurang efektif dalam menggunakan waktu.
	\item Mengalami kesulitan pada saat mengimplementasi kode program.
	\item Malu untuk bertanya pada dosen pembimbing.

\end{itemize}

\vspace{1cm}
\centering Bandung, \tanggal\\
\vspace{2cm} \nama \\ 
\vspace{1cm}

Menyetujui, \\
\ifdefstring{\jumpemb}{2}{
\vspace{1.5cm}
\begin{centering} Menyetujui,\\ \end{centering} \vspace{0.75cm}
\begin{minipage}[b]{0.45\linewidth}
% \centering Bandung, \makebox[0.5cm]{\hrulefill}/\makebox[0.5cm]{\hrulefill}/2013 \\
\vspace{2cm} Nama: \pembA \\ Pembimbing Utama
\end{minipage} \hspace{0.5cm}
\begin{minipage}[b]{0.45\linewidth}
% \centering Bandung, \makebox[0.5cm]{\hrulefill}/\makebox[0.5cm]{\hrulefill}/2013\\
\vspace{2cm} Nama: \pembB \\ Pembimbing Pendamping
\end{minipage}
\vspace{0.5cm}
}{
% \centering Bandung, \makebox[0.5cm]{\hrulefill}/\makebox[0.5cm]{\hrulefill}/2013\\
\vspace{2cm} Nama: \pembA \\ Pembimbing Tunggal
}
\end{document}

