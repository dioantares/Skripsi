\documentclass[a4paper,twoside]{article}
\usepackage[T1]{fontenc}
\usepackage[bahasa]{babel}
\usepackage{graphicx}
\usepackage{graphics}
\usepackage{float}
\usepackage[cm]{fullpage}
\pagestyle{myheadings}
\usepackage{etoolbox}
\usepackage{setspace} 
\usepackage{lipsum} 
\setlength{\headsep}{30pt}
\usepackage[inner=2cm,outer=2.5cm,top=2.5cm,bottom=2cm]{geometry} %margin
% \pagestyle{empty}

\makeatletter
\renewcommand{\@maketitle} {\begin{center} {\LARGE \textbf{ \textsc{\@title}} \par} \bigskip {\large \textbf{\textsc{\@author}} }\end{center} }
\renewcommand{\thispagestyle}[1]{}
\markright{\textbf{\textsc{AIF184001/AIF184002 \textemdash Rencana Kerja Skripsi \textemdash Sem. Genap 2022/2023}}}

\newcommand{\HRule}{\rule{\linewidth}{0.4mm}}
\renewcommand{\baselinestretch}{1}
\setlength{\parindent}{0 pt}
\setlength{\parskip}{6 pt}

\onehalfspacing
 
\begin{document}

\title{\@judultopik}
\author{\nama \textendash \@npm} 

%tulis nama dan NPM anda di sini:
\newcommand{\nama}{Dio Antares}
\newcommand{\@npm}{2017730003}
\newcommand{\@judultopik}{Pemanfaatan QR Code dalam Input Data Odoo, Studi Kasus: SIMU} % Judul/topik anda
\newcommand{\jumpemb}{1} % Jumlah pembimbing, 1 atau 2
\newcommand{\tanggal}{09/03/2023}

% Dokumen hasil template ini harus dicetak bolak-balik !!!!

\maketitle

\pagenumbering{arabic}

\section{Deskripsi}
Pada saat ini kebutuhan manusia terhadap teknologi sangatlah tinggi, dapat dilihat dalam kehidupan sehari-hari manusia tidak terlepas dari penggunaan alat teknologi, karena dalam penggunaan teknologi dapat berfungsi sebagai alat untuk mempermudah melakukan sesuatu. Kemajuan teknologi yang kian pesat pada era modern ini membawa berbagai dampak pada banyak aspek kehidupan, termasuk dalam suatu organisasi. Sistem Informasi Manajemen Umat (SIMU) adalah aplikasi milik Keuskupan Bandung, aplikasi ini bertujuan untuk mencatat data umat dan dinamikanya (contohnya adalah sakramen). Keuskupan Bandung memiliki sekitar 108.000 umat, plus umat Sibolga.

Dengan banyaknya jumlah umat yang terdapat dalam sistem informasi dan tidak menutup kemungkinan akan terus bertambah, maka akan dibuat sebuah sistem dengan memanfaatkan perkembangan teknologi pada saat ini. Salah satunya adalah judul skripsi penulis pada saat ini yaitu Pemanfaatan QR Code dalam Input Data Odoo, Studi Kasus: SIMU. Pemanfaatan QR Code ini bertujuan untuk mempermudah, mempercepat proses input data dan mengurangi kesalahan penulisan dalam input data, karena data yang diinput sudah berdasarkan penulisan umat itu sendiri.

Sebelum sistem ini dibuat, maka jika perlu ada data umat yang dimasukkan ke sistem informasi manajemen umat (SIMU), prosedurnya adalah sebagai berikut:
\begin{enumerate}
	\item Admin paroki memberikan blanko formulir data umat kepada umat.
	\item Umat mengisikan datanya ke dalam formulir tersebut secara tertulis.
	\item Formulir dikembalikan kepada admin paroki. 
	\item Admin paroki mengetikkan data yang dituliskan di atas formulir.
\end{enumerate}
Prosedur ini membutuhkan waktu yang lama dan kurang efisien, admin paroki memiliki kemungkinan untuk melakukan kesalahan dalam proses input data, karena admin paroki perlu untuk membaca ulang dan mengetikkan kembali data yang dituliskan di atas formulir kedalam sistem input data.

Pada skripsi ini yang berjudul Pemanfaatan QR Code dalam Input Data Odoo, Studi Kasus: SIMU, akan dibuat sebuah sistem yang dapat melakukan input data secara otomatis memanfaatkan QR Code pada input data Odoo, sehingga program yang dibuat ini akan menjadi dua aplikasi utama yaitu membuat halaman html sederhana (website) yang berisi form SIMU dan membuat sistem Odoo yang berisi data field yang menyerupai data umat SIMU dan sistem yang mampu memindai QR Code. Quick Response Code (QR Code) merupakan gambar dua dimensi yang memiliki kemampuan untuk menyimpan data. QR Code biasa digunakan untuk menyimpan data berupa teks, baik itu numerik, alfanumerik, maupun kode biner.

Program ini bertujuan agar umat dan admin paroki dapat lebih mudah dalam pengisian formulir, skripsi ini akan dirancang menggunakan Odoo. Odoo adalah aplikasi Enterprise Resource Planning open source berbasis Bahasa Python. Enterprise Resource Planning (ERP) adalah sebuah sistem informasi terintegrasi yang dapat mengakomodasi kebutuhan–kebutuhan informasi secara spesifik yang ada di perusahaan. Odoo adalah rangkaian aplikasi bisnis open source yang mencakup banyak kebutuhan, beberapa diantaranya adalah eCommerce, akuntansi, inventaris dan manajemen proyek.


\section{Rumusan Masalah}
\label{sec:rumusan}
Rumusan masalah yang akan dibahas di skripsi ini adalah sebagai berikut :
\begin{enumerate}
	\item Bagaimana membuat proses pencatatan data tidak perlu dilakukan secara manual?
	\item Bagaimana agar data yang telah dituliskan oleh umat dapat dipindai oleh sistem simulasi SIMU?
	\item Bagaimana agar data yang telah dituliskan oleh umat dapat diakses ke sistem simulasi SIMU?
\end{enumerate} 

\section{Tujuan}
\label{sec:tujuan}
Tujuan yang ingin dicapai dari penulisan skripsi ini sebagai berikut :
\begin{enumerate}
	\item Membangun halaman HTML yang yang responsif (terbaca mudah di ponsel) dan berisikan formulir sistem informasi manajemen umat (SIMU).
	\item Membangkitkan kode QR berdasarkan data yang telah diisi untuk nantinya dibaca oleh Odoo.
	\item Membangun sistem Odoo yang berisi data yang field-fieldnya menyerupai data umat dan sistem mampu memindai kode QR dari halaman formulir yang telah diisi oleh umat.
\end{enumerate}.

\section{Deskripsi Perangkat Lunak}
Perangkat lunak akhir yang akan dibuat memiliki fitur minimal sebagai berikut:
\begin{itemize}
	\item Halaman HTML yang berisikan form SIMU dan responsif (terbaca mudah di ponsel).
	\item Membuat sistem simulasi SIMU untuk keperluan percobaan.
		
\end{itemize}

\section{Detail Pengerjaan Skripsi}
Bagian-bagian pekerjaan skripsi ini adalah sebagai berikut :
\begin{enumerate}
	\item Melakukan studi mengenai komponen yang diperlukan untuk membuat Pemanfaatan QR Code dalam Input Data Odoo, Studi Kasus : SIMU
	\item Mempelajari struktur Odoo
	\item Membuat halaman HTML berisikan form SIMU dan menampilkan QR Code untuk dipindai. 
	\item Merancang sistem Odoo untuk SIMU
	\item Melakukan pengujian dan eksperimen
	\item Menulis bab 1-6 dokumen skripsi
\end{enumerate}

\section{Rencana Kerja}
Rincian capaian yang direncanakan di Skripsi 2 adalah sebagai berikut:
\begin{enumerate}
	\item Melakukan studi mengenai komponen yang diperlukan untuk membuat Pemanfaatan QR Code dalam Input Data Odoo, Studi Kasus : SIMU
	\item Mempelajari struktur Odoo
	\item Membuat halaman HTML berisikan form SIMU dan menampilkan QR Code untuk dipindai.
	\item Merancang sistem Odoo untuk SIMU
	\item Melakukan pengujian dan eksperimen
	\item Menulis bab 1-6 dokumen skripsi
\end{enumerate}

\vspace{1cm}
\centering Bandung, \tanggal\\
\vspace{2cm} \nama \\ 
\vspace{1cm}

Menyetujui, \\
\ifdefstring{\jumpemb}{2}{
\vspace{1.5cm}
\begin{centering} Menyetujui,\\ \end{centering} \vspace{0.75cm}
\begin{minipage}[b]{0.45\linewidth}
% \centering Bandung, \makebox[0.5cm]{\hrulefill}/\makebox[0.5cm]{\hrulefill}/2013 \\
\vspace{2cm} Nama: \makebox[3cm]{\hrulefill}\\ Pembimbing Utama
\end{minipage} \hspace{0.5cm}
\begin{minipage}[b]{0.45\linewidth}
% \centering Bandung, \makebox[0.5cm]{\hrulefill}/\makebox[0.5cm]{\hrulefill}/2013\\
\vspace{2cm} Nama: \makebox[3cm]{\hrulefill}\\ Pembimbing Pendamping
\end{minipage}
\vspace{0.5cm}
}{
% \centering Bandung, \makebox[0.5cm]{\hrulefill}/\makebox[0.5cm]{\hrulefill}/2013\\
\vspace{2cm} Nama: \makebox[3cm]{\hrulefill}\\ Pembimbing Tunggal
}
\end{document}

